%% AMS-LaTeX Created by Wolfram Mathematica 9.0 : www.wolfram.com

\documentclass{article}
\usepackage{amsmath, amssymb, graphics, setspace}
\usepackage{graphicx}

\usepackage[utf8]{inputenc}
\usepackage{spverbatim}
\usepackage{geometry}
 \geometry{
 a4paper,
 left=15mm,
 right=10mm,
 top=20mm,
 bottom=20mm,
 }

\DeclareMathSizes{10}{10}{10}{10}

\newcommand{\mathsym}[1]{{}}
\newcommand{\unicode}[1]{{}}

\newcounter{mathematicapage}
\begin{document}


\section{Ondas de sonido}
Una perturbación de pequeña amplitud en un fluido va a crear zonas de compresión y rarefacción que alternan en el tiempo y espacio en la dirección de propagación(el movimiento oscilatorio de las particulas se propaga como una onda y son ondas logitudinales:  las particulas se mueven en la misma dirección que la dirección de propagación)

\paragraph{Condiciones iniciales}
Consideramos un fluido ideal con las variables velocidad, densidad y presión(dependen de x y t) en el estado de equilibrio 
$v_{00} = 0, \rho_0 , p_0$ ($p_0, \rho_0$ constantes en el caso homogeneo, variando lentamente en espacio y/o tiempo en el caso inhomogeneo)
 y hacemos unos cambios en estas variables añadiendo unas perturbaciones:
$ v, \rho\prime, p\prime $ 
que  son muy pequeños de forma que se pueden omitir los términos de segundo orden (o mayores) en las ecuaciones de los fluidos.
Definimos 
\begin{description}  
 \item $c_s = \sqrt{\gamma \frac{p_0}{\rho_0}}$

 \item $c_s  =\big( \frac{\partial p}{\partial \rho}\big)_s  $

\end{description}  

Las variables de las equaciones de los fluidos devienen:

\begin{description}  
\item v
\item $\rho = \rho_0 + \rho\prime$
\item $p = p_0 + p\prime$

 
\end{description}  

con $|v|<<c_s, p\prime<<p_0 , \rho\prime << \rho_0$ 

\paragraph{Evolución}
Queremos estudiar la evolución en el tiempo de las perturbaciones: $p\prime, \rho\prime, v$
Las ecuaciones de los fluidos despues de linearizar devienen:
\begin{description}  
\item continuidad: $ \frac{\partial \rho\prime}{\partial t} + \rho_0 \nabla \cdot v = 0 $
\item movimiento: $ \frac{\partial v}{\partial t} + \frac{1}{\rho_0} \nabla p\prime = 0 $
\item energia:(El proceso es adiabático) $p\prime = c_s^{2} \rho\prime$

\end{description} 
Notaciones
\begin{description}  
\item N numero de dimensiones, $(x_i)_i$ vector $\in \mathbb{R}^N $, $f(x)=\big(f_i(x)\big)_i$ vector $\in \mathbb{R}^N $(como la velocidad), g(x) scalar $\in \mathbb{R}$
como la densidad, $p(x) = p(x)_{i,j}$ matriz $\in \mathbb{R}^N \times \mathbb{R}^N $ como $f_c$ de abajo
\item $\nabla \cdot f(x) = div(f) = \sum_i\frac{\partial f_i}{\partial x_i}$
\item $\nabla \cdot g(x) = div(g) = \sum_i\frac{\partial g}{\partial x_i}$
\item $\nabla f(x) = \big(\frac{\partial f_i}{\partial x_i}\big)_i$
\item $\nabla g(x) = \big(\frac{\partial g}{\partial x_i}\big)_i$
\item $\nabla \cdot p(x) = \big(\sum_j\frac{\partial p_{i,j}}{\partial x_j}\big)_i$
\item $x \cdot y = \sum_{i} x_i y_i$ para $\big(x_i\big)_i , \big(y_i\big)_i$ vectores
\end{description} 
 
\textbf{ De forma numérica} las ecuaciones de los fluidos linearizadas  de arriba se expresan: 
\begin{description}  
\item en el momento incial definimos:
\item $u_m = \rho$
\item $u_c = \rho v$
\item $u_e = \frac{p}{\gamma - 1} + \frac{1}{2}  \rho  |v|^2 $
\item En cada paso de tiempo
\begin{enumerate}  
\item se calculan los flujos definidos:
\begin{description}
\item $f_m = \rho v$
\item $f_c = (\rho v_i v_j)_{i,j} + p I_{N}$ 
\item matriz de dimension (N,N) N = numero de dimensiones del espacio y $I_N$ la matriz identidad
\item $f_e = u_e p v$
\end{description}  
\item y se recalculan $u_m u_e, u_c$ (aqui se pueden usar esquemas diferentes que se configuran en constants.py: lax-fr("lf") o first generation("fg")). 
Tambien en este paso se aplican las condiciones de contorno el tipo se condfigura en soundwave\_boundary\_conditions.py 
y se han implementado "repeat"y "refl" (en el caso se la esquema "fg" estas se pueden aplicar también en el paso 
intermedio dependiendo del parámetro bcStep de constants.py ("interm" o "final"))

\begin{description}
\item $\frac{du_m}{dt} = -\nabla \cdot f_m$
\item $\frac{du_e}{dt} = -\nabla \cdot f_e$
\item $\frac{du_c}{dt} = -\nabla \cdot f_c$
\end{description}  

\item y las variables v,p,$\rho$:
\begin{description}
\item $v = \frac{1}{\rho} u_c$
\item $p = (\gamma-1) (u_e - \frac{1}{\rho^2}{|u_c|^2})$
\item $\rho = u_m$
\end{description}  
\end{enumerate}  
\item dt depende de los valores de las variables en el momento que se calcula $\rho, p, v$, resolución (nint), y el tipo de esquema numérico usado (depende del parametro fcfl que es diferente para las esquema lax-fr y first generation (que se implemnetaron en la práctica))
 


\end{description}  

\textbf{de forma analítica}:

\begin{description}  



\item Después de hacer cálculos (y suponiendo el caso general inhomogeneo en tiempo y espacio : 
$\rho_0 = \rho_0(x,t),p_0 = p_0(x,t) \implies c_s = c_s(x,t)$ 
con  $p_0, \rho_0, c_s$  monótonas, variando bastante lentamente en el espacio y tiempo de tal forma que se pueden omitir terminos de segundo orden o mas en multiplicaciones de derivadas temporales o espaciales de estas y las perturbaciones) las perturbaciones de las variables verifican las ecuaciones: 

\item $\frac{\partial}{\partial t} \big(\frac{1}{c_s^{2}(x,t)} \frac{\partial p\prime}{\partial t}\big) = \nabla^{2} p\prime    $
\item $\frac{\partial}{\partial t} \big(\frac{1}{c_s^{2}(x,t)} \frac{\partial \rho\prime}{\partial t}\big) = \nabla^{2} \rho\prime    $
\item $\frac{\partial}{\partial t} \big(\frac{1}{c_s^{2}(x,t)} \frac{\partial \Phi}{\partial t}\big) = \nabla^{2} \Phi $ donde definimos  $v = \nabla \Phi$ considerando el fluido irrotacional
\item TODO: Estas relaciones serían válidas también cuando $p_0$ no es constante? - en la práctica consideramos $p_0$ constante y $\rho_0 = \rho_0(x)$
\item(
{\small{he mencionado este caso general con la variación de $\rho_0$ también  en el tiempo 
porque en la documentación que he leído se consideraba así}})

\end{description}  

\paragraph{Medio homogéneo}

\begin{description}  
\item Si la densidad y presión de equilibrio son constantes en tiempo y espacio ($\rho_0, p_0 constantes \implies c_s const$) las soluciones generales de:
\item $\frac{1}{c_s^{2}} \frac{\partial^{2} p\prime}{\partial t^{2}} = \nabla^{2} p\prime    $
\item $\frac{1}{c_s^{2}} \frac{\partial^{2} \rho\prime}{\partial t^{2}} = \nabla^{2} \rho\prime    $
\item $\frac{1}{c_s^{2}} \frac{\partial^{2} \Phi}{\partial t^{2}} = \nabla^{2} \Phi    $
\item  con $c_s$ const. son superposiciones de ondas planas de forma
\item $ \sum_{i} A_{\rho\prime_i} cos (x - csSign_i c_s t) $
\item $ \sum_{i} A_{p\prime_i} cos (x - csSign_i c_s t) $
\item $ \sum_{i} csSign_i A_{v_i} cos (x - csSign_i c_s t) $ 
\item con las amplitudes números reales verificando las ecuaciones:


\item $A_{p\prime_i} = c_s^{2} A_{\rho\prime_i}$ (de la ecuación de adiab.)
\item $A_{v_i} = \frac{1}{c_s \rho_0} A_{p\prime_i}$  (de la ecuación de movimiento)
\item y en los  casos $csSign_i = 1, A_{\rho\prime_i}, A_{p\prime_i}, A_{v_i} > 0$ y 
$csSign_i = -1, A_{\rho\prime_i}, A_{p\prime_i}, A_{v_i} < 0$  la onda viaja a la derecha
\item y cuando $csSign_i = -1, A_{\rho\prime_i}, A_{p\prime_i}, A_{v_i} > 0$ y 
$csSign_i = 1, A_{\rho\prime_i}, A_{p\prime_i}, A_{v_i} < 0$  la onda va a la izquierda

\item En la práctica pondrémos (cuando de generan las condiciones iniciales y cuando se calcula la solución analítica):
\item $A_{p\prime_i} = \gamma   p_0  A$
\item $A_{\rho\prime_i} = \rho_0  A$
\item $A_{v_i} = c_s  A$
\item con A (la amplitud de la perturbación) muy pequeña 

\end{description}  


\paragraph{Medio inhomogéneo independiente de tiempo} $p_0 = p_0(x),\rho_0 = \rho_0(x)\implies c_s = c_s(x)$ monótonas, variando lentamente..
\begin{description}  
\item $\frac{1}{c_s^{2}(x)} \frac{\partial^{2} p\prime}{\partial t^{2}} = \nabla^{2} p\prime    $
\item $\frac{1}{c_s^{2}(x)} \frac{\partial^{2} \rho\prime}{\partial t^{2}} = \nabla^{2} \rho\prime    $
\item $\frac{1}{c_s^{2}(x)} \frac{\partial^{2} \Phi}{\partial t^{2}} = \nabla^{2} \Phi    $
\item La solución no se puyede expresar mas como superposición de ondas planas (porque $c_s$ no es más constante ), pero análogo a esta solución del caso homogeneo de la onda plana:
 $p(x,t)=a cos(\phi(x))$ donde $\phi(x,t) = kx-\omega t$ con la amplitud a const y $\omega, k$ constantes verificando la relación de dispersión $\omega^{2} = c_s^2 k^2  $ intentamos  buscar soluciones de forma $a(x,t) e^{i\phi(x,t)}$ (aproximación WKB): 
 donde definimos 
\item $\omega(x,t) = -\frac{\partial \phi}{\partial t}$
\item $k(x,t) = -\nabla \phi$
\end{description}


\paragraph{Resolver la ecuación genérica}
\begin{description}  
\item $\frac{1}{c_s^{2}(x)} \frac{\partial^{2} p}{\partial t^{2}} = \nabla^{2} p  $
\item Reemplazando la solución WKB approx. $p(x,t) = a(x,t) e^{i \phi(x,t)}$ en la ecuación y con las definiciones de $\omega$ y k de arriba despues 
de hacer los cálculos 
y asumiendo  que las variaciones en la amplitud son muy pequeñas de forma que podemos omitir términos de segundo orden en las derivadas espaciales y temporales de a llegamos a: 
\begin{itemize}
\item $\omega^{2}(x,t) = c_s^{2}(x) k^2(x,t)$ (la relación de dispersión es válida de forma local)
\item y la ecuación de la evolución de la amplitud:
  $\frac{\partial a}{\partial t} + c_g \cdot \nabla a = -\frac{1}{2} \frac{a}{|k| c_s} (\frac{\partial \omega}{\partial t} + c_s^{2}  \nabla \cdot k) $
\end{itemize}
\item con $c_g = \frac{\partial \omega}{\partial k}$ 

\item de la relación de dispersión $\implies $ 
\item$\frac{\partial \omega}{\partial t} + c_g \cdot \nabla \omega = 0 $
\item$\frac{\partial k}{\partial t} + c_g \cdot \nabla k = -k \cdot \nabla  c_g $
\item$\frac{\partial \phi}{\partial t} + c_g \cdot \nabla \phi = 0 $

\item Usamos directamente la ecuación de conservación de energía para determinar la amplitud y no la de arriba (TODO):
\item$\frac{\partial E}{\partial t} + c_g \cdot \nabla E = -E \nabla \cdot c_g $

\end{description}


\paragraph{1D} 
($c_g = c_s$):
\begin{description}  
\item$\frac{\partial \omega}{\partial t} + c_s \frac{\partial \omega}{\partial x} = 0 $
\item$\frac{\partial k}{\partial t} + c_s \frac{\partial k}{\partial x} = -k \frac{\partial c_s}{\partial x} $
\item$\frac{\partial \phi}{\partial t} + c_s \frac{\partial \phi}{\partial x} = 0 $
\item $\frac{\partial a}{\partial t} + c_s \frac{\partial a}{\partial x} = -\frac{1}{2} \frac{a}{|k| c_s} (\frac{\partial \omega}{\partial t} + c_s^{2} \frac{\partial k}{\partial x}) $

\item$\frac{\partial E}{\partial t} + c_s \frac{\partial E}{\partial x} = -E \frac{\partial  c_s}{\partial x} $


\end{description}

Al largo de un rayo (similar a una trayectoria) $x_p(t)$ solución de :

\begin{description}  
\item$\frac{dx}{dt} = c_s $
\item$ x(0) = x_p $
\end{description}
las dependencias de x se transforman en dependencias de t reemplazando x por $x_p(t) $ 

y reemplazando en las ecuaciones de arriba obtenemos  las ecuaciones diferenciales:
\begin{description}  
\item $\frac{d\omega}{dt} = 0$
\item $\frac{d\phi}{dt} = 0$
\item $\frac{dk}{dt} = -k \frac{\partial c_s}{\partial x} $
\item $\frac{da}{dt} = -\frac{1}{2} \frac{a}{|k| c_s} (\frac{\partial \omega}{\partial t} + c_s^{2}  \frac{\partial c_s}{\partial x}) $
\item$\frac{dE}{dt} = -E \frac{\partial  c_s}{\partial x} $

\end{description}
al largo del rayo
Las ecuaciones ahora no tienen mas derivadas parciales (solo dependen de t) y se pueden integrar más fácil.
\begin{description}
\item De la primera se obtiene:  
\item $ \omega(x_p(t), t)  = constant $
\end{description}
Haciendo unos cálculos:
\begin{description}  
\item $\frac{dc_s}{dt} =  \frac{\partial c_s}{\partial x} \frac{\partial x}{\partial t} = \frac{\partial c_s}{\partial x}  c_s\implies
\frac{dk}{dt} = -k  \frac{1}{c_s} \frac{dc_s}{dt} \implies$
\item $\frac{dln(k)}{dt} = - \frac{dln(c_s)}{dt} \implies $
\item $ k(x_p(t), t)  c_s(x_p(t)) = constant $
\item de forma similar $ E(x_p(t), t)  c_s(x_p(t)) = constant $ 
\end{description}

\textbf{Relación entre las amplitudes}

\begin{description}  
\item Suponiendo que estas soluciones existen:
\item $p\prime = P e^{i\phi}$
\item $\rho\prime = R e^{i\phi}$
\item $v = V e^{i\phi}$
\item P,R,V  complejos
\end{description}

igual que en caso homogeneo :

\begin{description}  
\item de la relación de adiabaticidad: $|P(x,t)| = c_s^{2}(x) |R(x,t)|$
\item de la ecuación de movimiento: 
\item $\rho_0 \frac{\partial v}{\partial t} = -\nabla \rho\prime \implies -i \rho_0 \omega V e^{i\phi} = i k P e^{i\phi}$
\item despues de simplificar,  multiplicar cada lado con su conjugado(hay que expresar las amplitudes locales con el módulo porque pueden ser complejas):
\item  $|V|^2 = \frac{|P|^2}{\rho_0^{2} c_s^2} \implies$ 
\item $|V|= \frac{1}{c_s \rho_0} |P| = \frac{c_s}{p_0 \gamma} |P|$ 


\end{description}
\begin{description}  

\item De las relaciones entre las amplitudes $ \implies \frac{\rho_0 |V|^{2}}{2} = \frac{|P|^{2}} {2 \rho_0 c_s^{2}} $
\item $E = \frac{|P|^{2}}{\rho_0 c_s^2} $ (la suma de las 2)
\item $\implies |P(x_p(t),t)|^2 \rho_0^{-1}(x_p(t))c_s^{-1}(x_p(t)) = constant $
\item Suponemos que podemos escribir para los puntos al largo del rayo $x_p(t)$ :
\item $|P| = c_s^{k} \rho_0^{l} p_0^{m} C_{p_P}$
\item $|R| = c_s^{k} \rho_0^{l_1} p_0^{m_1} C_{p_R}$
\item $|V| = c_s^{k} \rho_0^{l_2} p_0^{m_2} C_{p_V}$
\item con $C_{p_P}, C_{p_R} y C_{p_V}$ constantes (que dependen del punto $x_p$)
\item de la ecuación de conservación de energía al largo del rayo $x_p(t)$:
 $|P(x_p(t),t)| = const * c_s^{\frac{1}{2}}(x_p(t)) \rho_0^{\frac{1}{2}}(x_p(t)) \implies$
\item  existen las relaciones: $ k = 2l - \frac{1}{2} = \frac{1}{2} - 2m$ y $2m + 2l = 1$

\item k se puede dejar como parámetro ya que $\gamma = const = c_s^{2} \rho_0^{-1} p_0^{-1} $ y multiplicando las ecuaciones con $\gamma^{p}$ 
el parámetro $k\rightarrow k+2p$ 
\item elegimos $k = l = \frac{1}{2}$ , $m = 0$ 
\item de las relaciones entre las amplitudes de arriba 
\item $\frac{|P|}{|R|} = c_s^2 = \rho_0^{l-l_1} p_0^{m-m_1} \frac{C_{P}}{C_{R}}$
\item Escribimos $c_s^2 = \gamma p_0 \rho_0^{-1}$ y la relación de arriba se tiene que cumplir $\forall x$ (porque consideramos $c_s, \rho_0 y p_0 $ dependiendo de x)
\item $\implies m - m_1 = 0$, $l-l_1 = 0$ y $\frac{C_P}{C_R} = \gamma$
\item $\frac{|P|}{|V|} = c_s \rho_0 = \gamma^{-\frac{1}{2}} p_0^{\frac{1}{2}} \rho_0^{\frac{1}{2}} = \rho_0^{l-l_2} p_0^{m-m_2} \frac{C_P}{C_V}$
\item $\implies  \frac{C_P}{C_V} =  \gamma^{-\frac{1}{2}}, l-l_2=\frac{1}{2}. m-m_2 = \frac{1}{2} $
\item $\implies m_1=-1, l_1 = \frac{3}{2}, l_2 = 0, m_2 = -\frac{1}{2}$
\item En conclusión: al largo del rayo $x_p(t)$ 
\item $|P| = c_s^{\frac{1}{2}} \rho_0^{\frac{1}{2}}  C_{P}$
\item $|R| = c_s^{\frac{1}{2}} \rho_0^{\frac{3}{2}} p_0^{-1} \gamma^{-1} C_{P}$
\item $|V| = c_s^{\frac{1}{2}}  p_0^{-\frac{1}{2}} \gamma^{\frac{1}{2}} C_{P}$

\item con $C_P$ constante que depende de $x_p$ ($C_P = |P(x_p,0)|  c_s^{-\frac{1}{2}} \rho_0^{-\frac{1}{2}}$ donde $|P(x_p,0)|$
es el valor de la amplitud de la perturbación de la presión en el momento t = 0 y el punto $x_p$ )

\end{description}

\section{Práctica}
\begin{description}
\item Definimos una amplitud A(en soundwave\_perturbation\_params.py) muy pequeña y una función periodica h en el intervalo $[x_0, x_f]$ (estos están en constants.py) con amplitud máxima 1 y creeamos las perturbaciones para que cumplan las relaciones entre las amplitudes de arriba:
\item  $p\prime(x,0) = \gamma p_0 h(x) $ 
\item  $\rho\prime(x,0) = \rho_0 h(x) $
\item  $v(x, t) =  c_s h(x) $
\item h periodica $\implies$ h se puede escribir como una superposición de ondas sinusoidales (componentes Fourier) que viajan a la derecha
\item Se pueden también definir superposiciones de funciones periodicas que pueden viajar en los 2 sentidos para generar las perturbaciones (condiciones iniciales) 
(definiendo A, csSign como se explicó arriba y el tipo de función para cada una  : en soundwave\_perturbation\_params.py)
\end{description}

\paragraph{Wave packet} Elegimos functionType = "wave\_packet" in soundwave\_perturbation\_params.py  (gauss wave packet: los parámetros $z_0,z_c,W$ definidos en sound\_wave\_packet\_params.py))
\begin{center}
 $h(x) = e^{- \frac{(x-x_c)^2}{W^2}} cos(2 \pi  k_0 \frac{x - x_0}{x_f - x_0}) $
\end{center}

\subsection{Medio homogéneo}

\paragraph{Evolución en el tiempo de las variables}
\begin{description}
\item $A = 3e-4$
\item Como las 3 variables: $p\prime, \rho\prime, v$  cumplen las mismas ecuaciones diferenciales consideramos $h_2(x,t)$ solución de la ecuación genérica:   
 $\frac{1}{c_s^{2}} \frac{\partial^{2} h_2}{\partial t^{2}} = \frac{ \partial^{2} h_2}{\partial x^2}   $ con $c_s$ constante  y condición inicial $h_2(x,0) = h(x)$
y luego la solución de las 3 variables es multiplcar esta con las amplitudes correspondientes

\item La solución analítica (y que está implemntada) es
$h_2(x,t) = h(x - c_s t)$
\item el gráfico de la  funcion inicial (en este caso el paquete de ondas gaussian) no cambia de forma solo se traslada a la derecha con la velocidad $c_s$

\begin{figure}[!ht]
 \centering
 \includegraphics[scale=0.2]{mainhom.png}
 \caption{\emph{El paquete de ondas en medio homogeneo en varios instantes del tiempo: boundary conditions: repeat}}
\end{figure}

\item si la resolución no es buena la amplitud calculada de forma numérica puede crecer. En este caso he tomado nint = 2048 (definido en constants.py)
y parece que la solución analítica dibujada por encima de la numérica no se desvía mucho de ella.
\end{description}


\paragraph{Fourier}
\textbf{Equivalencias de las definiciones}:

\begin{description}
\item La transformada Fourier de la función h se define  (omitiendo dominios):
$F_h(k) = \int_{- \infty}^{\infty}{h(x) e^{-i k x} dx } $
 
\item para $h(x)$ definida arriba

\item calculo con mathematica (de forma analítica) la siguiente integral
$f_1(k)=F_h(\frac{2\pi k}{x_f - x_0} ) = \int_{- \infty}^{\infty}{h(x) e^{-\frac{2 \pi i k x }{x_f - x_0}} dx } $

\item calculo con mathematica (de forma analítica) la transformada fourier de h con FourierTransform usando FourierParameters $0, -2\pi$  y obtengo $f_2(k)$

\item Compruebo de forma numerica y analítica que
$f_1(k) = f_2(\frac{k}{x_f - x_0})$

\item De la teoria sé que si defino los coeficientes: $c(k) = \frac{1}{x_f - x_0} f_1(k) $
\item  puedo escribir $f(x) = \sum_{k=-\infty}^{\infty} c_k e^{2 \pi i k \frac{x-x_0}{x_f - x_0}}$

\item quiero representar de forma analítica y numeríca $c(k) * (x_f-x_0)$
\item de forma analítica: calculo en mathematica $f_1(k)$ o obtengo $f2(k)$ y represento  $f_2(\frac{k}{x_f - x_0})$

\item observo que en python los coeficientes calculados (de forma numérica) con scipy.fft son $cp(k) = f_1(\frac{k}{x_f-x_0})$ asi que 
\item de forma numérica: calculo en python con scipy.fft cp(k) y  represento $cp(k* (x_f - x_0))* (x_f - x_0)$  
\end{description}

\textbf{Resultado}:
\begin{description}
\item después de hacer unos cálculos llego a 
\item $f_1(k) = \frac{W \sqrt{\pi}}{2} (g_1(k) + g_2(k)) $ con
\item $g_1(k) =  e^{-\frac{\pi \big[ \pi  W^2 (k + k_0)^2   + 2  i x_c(x_f - x_0)(k+ k_0)   - 2 k_0 i x_0 (x_f-x_0) \big]}{(x_f-x_0)^2}}$
\item $g_2(k) = e^{-\frac{\pi \big[\pi  W^2 (k - k_0)^2  +2 i x_c(x_f - x_0)(k - k_0)  + 2 k_0 i x_0 (x_f-x_0)\big]}{(x_f - x_0)^2}} $

\item Considerando $x_c$ = 0 las exponenciales $g_1$ y $g_2$ son  gaussianas  con w = $\frac{x_f - x_0}{\pi W}$ la primera centrada en $-k_0$ y la segunda en $k_0$
\item si calculamos el módulo de cada una  el modulo las constantes $ abs(exp( - 2 k_0 i x_0 (x_f-x_0))) = abs(exp( 2 k_0 i x_0 (x_f-x_0)))  = 1$
(corresponden al desplazamiento de fase por la elección de $x_c$(el centro de la función gauss) y $x_0$(el centro de la función cos ))
\item la amplitud  máxima de cada una es $\frac{W \sqrt{\pi}}{2}$ igual que se ve en el gráfico por estar bien separadas (con valores :  z0=3.100, zf = 7.400, k0 = 60, zc = 3.745, W = 0.050) : con rojo había hecho el plot de la función entera y con verde y azul de las 2 partes 

\item \begin{figure}[!ht] 
 \centering 
 \includegraphics[scale=0.5]{gauss.png} 
\end{figure} 

\end{description}
\textbf{Evolución de $f_1(k)$ en el tiempo}  $c_k * (x_f - x_0)$
\begin{description}
\item Si escribimos   $h_2(x,0) = h(x) = \sum_{k=-\infty}^{\infty}c_k e^{2 \pi i k \frac{x-x_0}{x_f-x_0}}$ 
\item $h_2(x,t) = h(x) = \sum_{k=-\infty}^{\infty}c_k e^{2 \pi i k \frac{x-c_s t-x_0}{x_f-x_0}}$ 
\item $c_s$ constante $\implies $ los coeficientes quedan los mismos 

\item \begin{figure}[!ht]
 \centering
 \includegraphics[scale=0.2]{fourhom.png}
 \caption{\emph{evolución de f1(k)}}
\end{figure}
\item Igual que se ve en el gráfico $f_1(k) $ casi no varía en el tiempo (se representó tambien la solución analítica para hacer la comparación).
La pequeña variación se debe a la resolución (la solución numérica decrece un poco)


\end{description}

\paragraph{Condiciones de contorno: reflexión}

\begin{description}
\item cambiando el parametro periodicType in soundwave\_boundary\_conditions.py  podemos  elegir las condiciones de contorno ("refl" o "repeat")
\item
\begin{figure}[!ht]
 \centering
 \includegraphics[scale=0.2]{reflhom.png}
 \caption{\emph{Evolución en el tiempo de las variables boundary conditions: refl }}
\end{figure}
\item TODO: la solución analítica con periodicType="refl" no está todavía implemnetada
\end{description}

\subsection{Medio inhomogéneo}
$p_0 $ const y $\rho_0(x) definida:$
\begin{center}
	$\rho_{00} + \frac{\rho_{01}-\rho{00}}{2}\big[ 1 + tanh(\frac{z-z_e}{w_e})\big] $
\end{center}

({\small las configuraciones del medio estan en soundwave\_medium\_params.py mediumType = "homog" o "inhomog" , $rho_{00}$ y 
los demás parámetros del caso inhomogéneo: $z_e, w_e, \rho_{01}$ })

\newpage
\paragraph{$\rho_0$ decrece }($\rho_{00} > \rho_{01} $)




\begin{description}

\item Evolución de la presión, densidad, velocidad, $rhoCurve = \frac{\rho- \rho_0}{\rho_0} $)
\item \begin{figure}[!ht]
 \centering
 \includegraphics[scale=0.2]{maininhom1.png}
 \caption{\emph{presion, densidad, velocidad}}
\end{figure}

\item \begin{figure}[!ht]
 \centering
 \includegraphics[scale=0.2]{rhocinhom1.png}
 \caption{\emph{rhoCurve}}
\end{figure}

\item Las amplitudes de las perturbaciones de presión , densidad y velocidad  no son mas constantes al largo de los rayos como en el caso homogeneo igual que se ve en el gráfico de arriba, la amplitud de la presion y rhoCurve decrece y la de la velocidad crece (las time derivative  no son mas 0 como en el caso homogeneo)
\item En el caso homogeneo la solución para $\frac{1}{c_s^{2}} \frac{\partial^{2} h_2}{\partial t^{2}} = \frac{ \partial^{2} h_2}{\partial x^2}   $ con $c_s$ constante  y condición inicial $h_2(x,0) = h(x)$ en terminos de rayos se puede escribir:
\item al largo del rayo $x_p(t)$ (solución de $\frac{dx}{dt} = c_s, x_p(0) = x_p$) $\frac{dh_2}{dt} = 0$ (time derivative)
equivalente a $h_2(x_p(t), t) = const = h(x_p)$  

\end{description}

\newpage

\paragraph{Ray tracing}

\begin{description}
\item Marcamos en el los gráficos un punto $x_p = x_c$(el centro del paquete) y miramos la evolución en el tiempo desplazandolo en cada paso de tiepo con $c_s dt$. el punto se queda en el centro del paquete. Eso esperabamos porque los rayos son soluciones 
$x_p(t)$ de $\frac{dx}{dt} = c_g, x_p(0) = x_p$ y las amplitudes de las perturbaciones de presion, densidad y velocidad varian lentamente al largo de ellos  
\item De las relaciones de  la evolucion teoretica de las amplitudes: en el caso cuando $ p_0$ constante $\implies$  
\item $|P(x_p(t),t)| c_s^{\frac{1}{2}}(x_p(t),t) = constant $ al largo del rayo $x_p(t)$
\item y las otras 2 se calculan de las relaciones : $\frac{|P|}{|V|} = c_s \rho_0 , \frac{|P|}{|V|} = c_s^2$
\item en 1D $c_s=c_g$
\item la ecuacion de la variación de las amplitudes al largo de los rayos es válida para cualquier $x_p$

\item Sabemos de la teoria que al largo del rayo $x_p(t)$ $k(x_p(t),t) * c_s(x_p(t)) = const$ 
\item Comprobamos la solución numérica: calculamos $k_c$ el k para cual el valor de los coeficientes fourier $c(k)$ calculados con python 
de forma numerica con scipy.fft para el valor de la presión en el momento t
\item miramos la evolución en el tiempo de $k_c * c_s(x_p(t)) $ que debería ser constante = $k_0 c_s(x_p)$


\item \begin{figure}[!ht]
 \centering
 \includegraphics[scale=0.5]{kcinhom1.png}
 \caption{\emph{kc * cs(xp(t))}}
\end{figure}

\item \begin{figure}[!ht]
 \centering
 \includegraphics[scale=0.2]{pfourinhom1.png}
 \caption{\emph{f1(k) para la presion}}
\end{figure}
\item Deberia ser constante, pero las 

\item \begin{figure}[!ht]
 \centering
 \includegraphics[scale=0.2]{wwinhom1.png}
 \caption{\emph{ww}}
\end{figure}

\item Deberia ser constante, pero las 
\end{description}

\begin{description}

\item Las amplitudes se calculan al largo del rayo  $x_p(t)$ (la curva solución para  $\frac{dx}{dt} = c_s, x(0) = x_p$): $C_P$ es una constante que depende de los valores en el punto inicial del rayo ($x_p$): $C_P = c_s^{-\frac{1}{2}}(x_p) , \rho_0^{\frac{1}{2}}(x_p)$

\item En la práctica he calculado la solución analítica(los valores de las amplitudes $|P|, |R|, |V|$ ) al largo del rayo $x_p(t)$:
\item $|P(x_p(t),t)| = c_s^{-\frac{1}{2}}(x_p(t)) c_s^{\frac{1}{2}}(x_p) P(x_p,0)  $
\item las amplitudes de las demás perturbaciones se calculan reemplazando directamente en las relaciones entre las amplitudes (cuando se representa la solución analítica), 
\item En el caso de la práctica queremos representar las amplitudes como $const * cs^{\alpha} $ reemplazamos $\rho_0$
\item Para ver si el paquete se adapta a la relación representamos las curvas:
\item para la presión: $ c_s^{-\frac{1}{2}}(x) c_s^{\frac{1}{2}}(x) P(x_0,0) $
\item densidad: $ c_s^{-\frac{3}{2}}(x) c_s^{\frac{3}{2}}(x_0) R(x_0,0)  $ 
\item velocidad: $ c_s^{\frac{1}{2}}(x) c_s^{-\frac{1}{2}}(x_0) V(x_0,0)  $ 
\item eligiendo $x_0$ los 2 puntos donde  los valores de las amplitudes de las perturbación en el momento inicial tienen el mínimo y máximo 


\begin{figure}[!ht]
 \centering
 \includegraphics[scale=0.2]{mainInhom1.png}
 \caption{\emph{Perfil radial de las estrellas de referencia para la alineación}}
\end{figure}

\begin{figure}[!ht]
 \centering
 \includegraphics[scale=0.2]{rhocInhom1.png}
 \caption{\emph{Perfil radial de las estrellas de referencia para la alineación}}
\end{figure}
 
\end{description}

\end{document}



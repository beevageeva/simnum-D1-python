%% AMS-LaTeX Created by Wolfram Mathematica 9.0 : www.wolfram.com

\documentclass{article}
\usepackage{amsmath, amssymb, graphics, setspace}
\usepackage{graphicx}

\usepackage[utf8]{inputenc}
\usepackage{spverbatim}
\usepackage{geometry}
 \geometry{
 a4paper,
 left=15mm,
 right=10mm,
 top=20mm,
 bottom=20mm,
 }



\newcommand{\mathsym}[1]{{}}
\newcommand{\unicode}[1]{{}}

\newcounter{mathematicapage}
\begin{document}


\section{Ondas de sonido}
Una perturbación de pequeña amplitud en un fluido va a crear zonas de compresión y rarefacción que alternan en el tiempo y espacio en la dirección de propagación(el movimiento oscilatorio de las particulas se propaga como una onda y son ondas logitudinales:  las particulas se mueven en la misma dirección que la dirección de propagación)

\paragraph{Condiciones iniciales}
Consideramos un fluido ideal con las variables en el estado de equilibrio $v_0 = 0, \rho_0 , p_0$ y hacemos unos cambios en estas variables que  son muy pequeños de forma que se pueden omitir los términos de segundo orden (o mayores) en las ecuaciones de los fluidos
Definimos $c_s = \big( \frac{\partial p}{\partial \rho}\big)_s = \sqrt{\gamma \frac{p_0}{\rho_0}}$
Las variables de cuales queremos estudiar la evolución devienen:

\begin{description}  
\item v
\item $\rho = \rho_0 + \rho_0\prime$
\item $p = p_0 + p\prime$

 
\end{description}  

con $|v|<<c_s, p\prime<<p , \rho\prime << \rho_0$ 

Las ecuaciones de los fluidos despues de linearlizar devienen:
\begin{description}  
\item continuidad: $ \frac{\partial \rho\prime}{\partial t} + \rho_0 \nabla \cdot v = 0 $
\item Euler: $ \frac{\partial v}{\partial t} + \frac{1}{\rho_0} \nabla p\prime = 0 $

\end{description}  

\begin{description}  
\item El proceso es adiabático: $p\prime = c_s^{2} \rho\prime$

\item Después de hacer los cálculos (y suponiendo el caso general inhomogeneo en tiempo y espacio : $\rho_0 = \rho_0(x,t) \implies c_s = c_s(x,t)$ con la densidad de equilibrio variando bastante lentamente en el espacio y tiempo de tal forma que se pueden omitir terminos de segundo orden o mas en multiplicaciones de derivadas temporales o espaciales de estas y las perturbaciones) las perturbaciones de las variables verifican las ecuaciones: 

\item $\frac{\partial}{\partial t} \big(\frac{1}{c_s^{2}(x,t)} \frac{\partial p\prime}{\partial t}\big) = \nabla^{2} p\prime    $
\item $\frac{\partial}{\partial t} \big(\frac{1}{c_s^{2}(x,t)} \frac{\partial \rho\prime}{\partial t}\big) = \nabla^{2} \rho\prime    $
\item $\frac{\partial}{\partial t} \big(\frac{1}{c_s^{2}(x,t)} \frac{\partial \Phi}{\partial t}\big) = \nabla^{2} \Phi $ donde definimos  $v = \nabla \Phi$ considerando el fluido irrotacional
\item Si la densidad de equilibrio no varia en el tiempo($\rho_0 = \rho_0(x)$):
\item $\frac{1}{c_s^{2}(x)} \frac{\partial^{2} p\prime}{\partial t^{2}} = \nabla^{2} p\prime    $
\item $\frac{1}{c_s^{2}(x)} \frac{\partial^{2} \rho\prime}{\partial t^{2}} = \nabla^{2} \rho\prime    $
\item $\frac{1}{c_s^{2}(x)} \frac{\partial^{2} \Phi}{\partial t^{2}} = \nabla^{2} \Phi    $
\end{description}  

\paragraph{Medio homogeneo}

\begin{description}  
\item Si la densidad de equilibrio es constante en tiempo($\rho_0, c_s constantes$) y espacio las soluciones generales son superposiciones de ondas planas de forma:
\item $ \sum_{i} A_{\rho\prime_i} cos (x - c_s t) $
\item $ \sum_{i} A_{p\prime_i} cos (x - c_s t) $
\item $ \sum_{i} csSign_i A_{v_i} cos (x - csSign_i c_s t) $ donde $csSign_i = 1 $ si la onda viaja a la derecha y -1 si va a la izquierda
\item con las amplitudes números reales verificando las ecuaciones:
$A_{p\prime_i} = c_s^{2} A_{\rho\prime_i}$
$A_{v_i} = \frac{1}{c_s \rho_0} A_{p_i}$
\end{description}  


\paragraph{Medio inhomogeneo} $\rho_0 = \rho_0(x)$
La solución no se puyede expresar mas como superposición de ondas planas (porque $c_s$ no es más constante ), pero análogo a esta solución intentamos  buscar soluciones de forma $a(x,t) e^{i\phi(x,t)}$ (aproximación WKB):
 donde definimos 
\begin{description}  
\item $\omega(x,t) = -\frac{\partial \phi}{\partial t}$
\item $k(x,t) = -\nabla \phi$
\end{description}
porque análogo al caso homogeneo (considerando sin perder la generlaidad solo una onda y no una superposición) donde $\phi(x,t) = kx-\omega t$ con $\omega, k$ constantes verificando la relación de dispersión $\omega^{2} = c_s^2 k^2  $  

\textbf{Relación entre las amplitudes}

\begin{description}  
\item Suponiendo que estas soluciones existen:
\item $p\prime = P e^{i\phi}$
\item $\rho\prime = R e^{i\phi}$
\item $v = V e^{i\phi}$
\item P,R,V  complejos
\end{description}

igual que en caso homogeneo :

\begin{description}  
\item de la relación de adiabaticidad: $|P(x,t)| = c_s^{2}(x) |R(x,t)|$
\item de la ecuación de movimiento: 
\item $\rho_0 \frac{\partial v}{\partial t} = -\nabla \rho\prime \implies -i \rho_0 \omega V e^{i\phi} = i k P e^{i\phi}$
\item despues de simplificar,  multiplicar cada lado con su conjugado(hay que expresar las amplitudes locales con el módulo porque pueden ser complejas):
\item  $|V|^2 = \frac{|P|^2}{\rho_0^{2} c_s^2}$ ($|V(x,t)|= \frac{1}{c_s \rho_0} |P(x,t)| = \frac{c_s(x)}{p_0 \gamma} |P(x,t)|$) 


\end{description}

\paragraph{Resolver la ecuación genérica}
\begin{description}  
\item $\frac{1}{c_s^{2}(x)} \frac{\partial^{2} p\prime}{\partial t^{2}} = \nabla^{2} p\prime    $
\item Reemplazando la solución WKB approx. $p(x,t) = a(x,t) e^{i \phi(x,t)}$ en la ecuación y con las definiciones de $\omega$ y k de arriba despues de hacer los cálculos 
\item $\omega^{2}(x,t) = c_s^{2}(x) k^2(x,t)$
\item definiendo $c_g = \frac{\partial \omega}{\partial k}$ :
\item $\frac{\partial a}{\partial t} + c_g \cdot \nabla a = -\frac{1}{2} \frac{a}{|k| c_s} (\frac{\partial \omega}{\partial t} + c_s^{2} \cdot \nabla k) $

\item $\implies $ 
\item$\frac{\partial \omega}{\partial t} + c_g \cdot \nabla \omega = 0 $
\item$\frac{\partial k}{\partial t} + c_g \cdot \nabla k = -k \cdot \nabla c_g $

\item Usamos directamente la ecuación de conservación de energía para determinar la amplitud:
\item$\frac{\partial E}{\partial t} + c_g \cdot \nabla E = -E \nabla \cdot c_g $

\end{description}


\textbf{1D} ($c_s = c_g$):
\begin{description}  
\item $\frac{1}{c_s^{2}(x)} \frac{\partial^{2} p\prime}{\partial t^{2}} = \nabla^{2} p\prime    $
\item$\frac{\partial k}{\partial t} + c_s \frac{\partial k}{\partial x} = -c_s $
\item$\frac{\partial \omega}{\partial t} + c_s \frac{\partial \omega}{\partial x} = 0 $
\item$\frac{\partial k}{\partial t} + c_s \frac{\partial k}{\partial x} = -k \frac{\partial c_s}{\partial x} $
\item$\frac{\partial E}{\partial t} + c_s \frac{\partial E}{\partial x} = -E \frac{\partial  c_s}{\partial x} $


\end{description}

Al largo de un rayo definido por:

\begin{description}  
\item$\frac{dx}{dt} = c_s $
\item$ x_p(t) , x_p(0) = x_p $
\end{description}

las funciones:
\begin{description}  
\item $\omega_p(t) = \omega(x_p(t), t)$
\item$k_p(t) = k(x_p(t), t)$
\item$a_p(t) = a(x_p(t), t)$
\end{description}

verifican las ecuaciones
\begin{description}  
\item $\frac{d\omega_p}{dt} = 0$
\item $\frac{dk_p}{dt} = -k_p \frac{\partial c_s}{\partial x} (x_p(t))$
\item $\frac{dc_s}{dt}  =  \frac{\partial c_s}{\partial x}(x_p(t)) c_s$
\end{description}

\begin{description}  
\item $\frac{dk_p}{dt} = -k_p  \frac{1}{c_s} \frac{dc_s}{dt} \implies$
\item $\frac{dln(k_p)}{dt} = - \frac{dln(c_s)}{dt} \implies $
\item $ k(x_p(t), t)  c_s(x_p(t)) = constant $
\item de forma similar $ E(x_p(t), t)  c_s(x_p(t)) = constant $ 
\item $ \omega(x_p(t), t)  = constant $
\end{description}

Ecuación de la energía:
\begin{description}  

\item $\frac{\rho_0 |V|^{2}}{2} = \frac{|P|^{2}} {2 \rho_0 c_s^{2}} $
\item $E = \frac{|P|^{2}}{\rho_0 c_s^2} $ (la suma de las 2)

\end{description}

\begin{description}  
\item $|P(x_p(t),t)| c_s^{\frac{1}{2}} = constant $
\end{description}



\section{Transf fourier}

Salida de mathematica de la integral de la transf fourier:

\begin{spverbatim}

$Assumptions = {Element[{k0,z0,zf,zc,W}, Reals], k0>0, z0>0, zf >0 , zf >0, zc >0, W>0 }
Print[FullSimplify[Integrate[Exp[- (z-zc)^2 / W^2] Cos[2 Pi  k0 (z - z0)/ (zf - z0)] Exp[- 2 Pi I m z / (zf - z0)],  {z, -Infinity, Infinity} ]]]

\end{spverbatim}

$
f1(m)=F( \frac{2\pi m}{z_f - z_0} ) = \int_{- \infty}^{\infty}{e^{-\frac{(z-z_c)^2}{W^2}} cos(\frac{2 \pi k_0 (z-z_0)}{z_f - z_0}) e^{-\frac{2 \pi i m z }{z_f - z_0}} dz } = $

$\frac{1}{2} e^{-\frac{\pi  \left(\text{$k_0$}^2 \pi  W^2+m \left(m \pi  W^2-2 i \text{$z_c$} (\text{$z_0$}-\text{$z_f$})\right)+2 \text{$k_0$} \left(m \pi  W^2+i
(\text{$z_0$}+\text{$z_c$}) (\text{$z_0$}+\text{$z_f$})\right)\right)}{(\text{$z_0$}-\text{$z_f$})^2}} \left(e^{\frac{4 i \text{$k_0$} \pi  \text{$z_0$} (\text{$z_c$}+\text{$z_f$})}{(\text{$z_0$}-\text{$z_f$})^2}}+e^{\frac{4
\text{$k_0$} \pi  \left(m \pi  W^2+i \left(\text{$z_0$}^2+\text{$z_c$} \text{$z_f$}\right)\right)}{(\text{$z_0$}-\text{$z_f$})^2}}\right) \sqrt{\pi } W$


Salida de FourierTransform mathematica con FourierParameters 0, -2$\pi$ 

\begin{verbatim}

$Assumptions = {Element[{k0,z0,zf,zc,W}, Reals], k0>0, z0>0, zf >0 , zf >0, zc >0, W>0 }
h[z_, k0_, z0_, zf_, zc_, W_]:= Exp[-(z-zc)^2/W^2] Cos[2 Pi k0 (z - z0) / (zf - z0) ]
Print[FullSimplify[FourierTransform[h[z, k0, z0, zf, zc, W], z, k, FourierParameters->{0,-2 Pi}]] ]

\end{verbatim}
$f2(k)=$

$\frac{1}{2} \left(e^{-\frac{\pi  \left(\text{$k_0$}^2 \pi  W^2-2 \text{$k_0$} \left(k \pi  W^2-i \text{$z_0$}+i \text{$z_c$}\right) (\text{$z_0$}-\text{$z_f$})+k
\left(k \pi  W^2+2 i \text{$z_c$}\right) (\text{$z_0$}-\text{$z_f$})^2\right)}{(\text{$z_0$}-\text{$z_f$})^2}}+ \\
e^{-\frac{\pi  \left(\text{$k_0$}^2 \pi  W^2+2 \text{$k_0$}
\left(k \pi  W^2-i \text{$z_0$}+i \text{$z_c$}\right) (\text{$z_0$}-\text{$z_f$})+k \left(k \pi  W^2+2 i \text{$z_c$}\right) (\text{$z_0$}-\text{$z_f$})^2\right)}{(\text{$z_0$}-\text{$z_f$})^2}}\right)
\sqrt{\pi } W$

En esta reemplazo k = k / (zf - z0) y los graficos salen iguales ($f1(k) = f2(\frac{k}{z_f - z0})$)


Ademas la salida de:
\begin{spverbatim}
$Assumptions = {Element[{k0,z0,zf,zc,W}, Reals], k0>0, z0>0, zf >0 , zf >0, zc >0, W>0 }

f1[m_]:=((E^(((4*I)*k0*Pi*z0*(zc + zf))/(z0 - zf)^2) + E^((4*k0*Pi*(m*Pi*W^2 + I*(z0^2 + zc*zf)))/(z0 - zf)^2))*Sqrt[Pi]*W)/
  (2*E^((Pi*(k0^2*Pi*W^2 + m*(m*Pi*W^2 - (2*I)*zc*(z0 - zf)) + 2*k0*(m*Pi*W^2 + I*(z0 + zc)*(z0 + zf))))/(z0 - zf)^2))
f2[k_]:=((E^(-((Pi*(k0^2*Pi*W^2 - 2*k0*(k*Pi*W^2 - I*z0 + I*zc)*(z0 - zf) + k*(k*Pi*W^2 + (2*I)*zc)*(z0 - zf)^2))/(z0 - zf)^2)) +
      E^(-((Pi*(k0^2*Pi*W^2 + 2*k0*(k*Pi*W^2 - I*z0 + I*zc)*(z0 - zf) + k*(k*Pi*W^2 + (2*I)*zc)*(z0 - zf)^2))/(z0 - zf)^2)))*Sqrt[Pi]*W)/2

Print[FullSimplify[f1[k]-f2[k/(zf-z0)]]]

\end{spverbatim}

es 0


Elijo f2 forma para simplificar (después de hacer los gráficos de los modulos de los valores de la función , tal como imaginaba la primera exponencial corresponde a la gaussiana de las frecuancias negativas y la segunda de las frequencias positivas):


$f1(k) = f2(\frac{k}{(z_f - z_0)}) = $

$ \frac{W \sqrt{\pi}}{2}  (e^{-\frac{\pi (k_0^2 \pi  W^2 + 2 k_0 k \pi W^2 + 2 k_0 i  (z_c - z_0) (z_f-z_0)+k^2 \pi W^2 +
2 i k z_c (z_f - z_0))}{(z_0-z_f)^2}}+e^{-\frac{\pi  (k_0^2 \pi  W^2 - 2 k_0 k \pi  W^2 - 2 k_0 i (z_c - z_0) (z_f-z_0)+k^2 \pi  W^2+2 i k z_c (z_f - z_0))}{(z_0-z_f)^2}}) =
$

$ \frac{W \sqrt{\pi}}{2}  (e^{-\frac{\pi \big[ \pi  W^2 (k + k_0)^2   + 2 k_0 i z_c(z_f - z_0)   - 2 k_0 i z_0 (z_f-z_0) +
2 i k z_c (z_f - z_0)\big]}{(z_0-z_f)^2}}+e^{-\frac{\pi \big[\pi  W^2 (k - k_0)^2  -2 k_0 i z_c(z_f - z_0)  + 2 k_0 i z_0 (z_f-z_0)+2 i k z_c (z_f - z_0)\big]}{z_0-z_f)^2}}) =
$

$ \frac{W \sqrt{\pi}}{2}  (e^{-\frac{\pi \big[ \pi  W^2 (k + k_0)^2   + 2  i z_c(z_f - z_0)(k_0 + k)   - 2 k_0 i z_0 (z_f-z_0) \big]}{(z_0-z_f)^2}}+e^{-\frac{\pi \big[\pi  W^2 (k - k_0)^2  +2 i z_c(z_f - z_0)(k - k_0)  + 2 k_0 i z_0 (z_f-z_0)\big]}{(z_0-z_f)^2}}) =
$

Considerando $z_c$ = 0 
las exponenciales son  gaussianas  con w = $\frac{z_f - z_0}{\pi W}$ la primera centrada en $-k_0$ y la segunda en $k_0$
y cuando calculamos el modulo las constantes $ abs(exp( - 2 k_0 i z_0 (z_f-z_0))) = abs(exp( 2 k_0 i z_0 (z_f-z_0)))  = 1$
y la amplitud queda $\frac{W \sqrt{\pi}}{2}$ igual que se ve en el gráfico (con valores :  z0=3.100, zf = 7.400, k0 = 60, zc = 3.745, W = 0.050) : con rojo había hecho el plot de la función entera y con verde y azul de las 2 partes al principio para estar segura que correspondían a las 2 partes

\begin{figure}[!ht] 
 \centering 
 \includegraphics[scale=0.5]{gauss.png} 
\end{figure} 
 

\end{document}


